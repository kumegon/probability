\documentclass[a4j,11pt]{jarticle}
\usepackage{url}
\usepackage[dvipdfmx]{graphicx}
\usepackage{here}
\usepackage{amssymb}
\usepackage{amsmath,amssymb}
\usepackage[dvipdfmx]{hyperref}
\usepackage{enumerate}
\usepackage{mathtools}
\graphicspath{{picture/}}
\usepackage{ascmac}
\usepackage{bm}
\usepackage{amsthm}
\usepackage{algorithm}
\usepackage{algorithmic}
\usepackage{comment}

\setlength{\textwidth}{1.1\textwidth}
\setlength{\oddsidemargin}{-3pt}
\setlength{\evensidemargin}{\oddsidemargin}
\setlength{\topmargin}{-10mm}
\setlength{\headheight}{0mm}
\setlength{\headsep}{0mm}
\begin{comment}
\usepackage{listings,jlisting}
\lstset{language=c,
  basicstyle=\ttfamily\scriptsize,
  commentstyle=\textit,
  classoffset=1,
  keywordstyle=\emphseries,
  frame=tRBl,
  framesep=5pt,
  showstringspaces=false,
  numbers=left,
  stepnumber=1,
  numberstyle=\tiny,
  tabsize=2
}
\end{comment}


\theoremstyle{definition}
\renewcommand{\proofname}{\bf{証明}}
\newtheorem{theorem}{定理}
\newtheorem*{theorem*}{定理}
\newtheorem{definition}[theorem]{定義}
\newtheorem*{definition*}{定義}
\newtheorem{lemma}[theorem]{補題}
\newtheorem*{lemma*}{補題}
\newtheorem{corollary}[theorem]{系}
\newtheorem*{corollary*}{系}
\newtheorem{proposition}[theorem]{命題}
\newtheorem*{proposition*}{命題}
\newtheorem{nature}[theorem]{性質}
\newtheorem{remark}[theorem]{注意}

\newcommand{\argmax}{\mathop{\rm arg~max}\limits}
\newcommand{\argmin}{\mathop{\rm arg~min}\limits}
\newcommand{\DtoR}[2]{{#1 \rightarrow #2}}
\newcommand{\dtor}[2]{{#1 \mapsto #2}}
\newcommand{\prox}[1]{{{\rm prox}_{#1}}}
\newcommand{\inprod}[2]{{\langle #1,#2 \rangle}}

\makeatother
\setcounter{secnumdepth}{4}


\title{舟木確率論ゼミ}
\author{M1 久米啓太}
\date{\today}


\begin{document}
\maketitle

\section*{命題2.25の証明の補足}
1)-3)を満たす
$\mathbb{R}$
上の関数
$F$
が与えられた時, (2.2)を満たすような
$(\mathbb{R},\mathcal{B}(\mathbb{R}))$
上の確率測度
$\mu$
が一意的に定まることの証明をする.\ 

$Y(w):\left( 0,1\right)\to \mathbb{R}:w\mapsto \sup \left\{y \in \mathbb{R} | F(y) < w \right\}, \ w \in \left(0,1 \right)$
とする.
集合
$A$
に対する
$y^\star = \sup A$
は以下を満たす
$\mathbb{R}$
の元である.\ 
\begin{itemize}
  \item 
    任意の
    $y \in A$
    に対して,\ 
    $y \leq y^\star$.
  \item
    任意の
    $\epsilon > 0$
    に対して,\ ある
    $y \in A$
    が存在して,\ 
    $y^\star - \epsilon < y$.
\end{itemize}
教科書では
\begin{quote}
$Y$
は
$F$
の左連続な逆関数
\end{quote}
であると書かれているが,\ 
$F$
に逆関数が存在するとは限らないことに注意する.\ 
なぜならば,\ 1)-3)の条件だけでは
$F$
が全単射であることを保証できないからである.\ 
$F$
が全単射であるためには,\ 狭義単調増加,\ 連続であればよい.\ 
$Y$
が左連続であるかは確かめることができなかった.\ 

\paragraph*{}
任意の
$x\in \mathbb{R}$
に対し,\ 
$\left\{ w \in \left( 0,1\right) | Y(w) \leq x \right\} = \left\{ w\in \left( 0,1\right) | w\leq F(x) \right\}$
が成り立つことを示す.\ 
左辺を
$A(x)$
右辺を
$B(x)$
と書くことにする.\ 

まず,\ 
$A(x) \subset B(x)$
を示す.\ 
$w \in A(x)$
とする.\ 
ここで,\ 
$w > F(x)$
であると仮定する.\ 
$F$
の右連続性から,\ ある
$\delta > 0$
に対して,\ 
$w > F(x + \delta)$
となる.\ 
よって,\ 
$Y(w) \geq x + \delta > x$
となる.\ 
これは
$w \in A(x)$
,\ つまり
$Y(w) \leq x$
であることに矛盾するので,\ 
$w \leq F(x)$
である.\ 
以上より,\ 
$w \in B(x)$,\ 
$A(x) \subset B(x)$.\ 

次に,\ 
$A(x) \supset B(x)$
を示す.\ 
$w \in B(x)$
とすると,\ 
$w \leq F(x)$
である.\ 
$x \not\in \left\{ y \in \mathbb{R} | F(y) < w\right\}$
より,\ 
$Y(w) \leq x$
である.\ 
以上より,\ 
$w \in A(x),\ $
$A(x) \supset B(x)$.\ 

\begin{equation}
  B(x) = \left\{ w  \in\left( 0,1\right) | w \leq F(x)\right\} = 
  \begin{cases}
    \left( 0, F(x)\right] & {\rm if } \ F(x) > 0 \\
    \emptyset & {\rm otherwise}
  \end{cases}
\end{equation}
なので,\ 明らかに
$B(x)$
は可測集合である.\ 
よって,\ 任意の
$x\in \mathbb{R}$
において
$A(x)$
も可測集合であるので,\ 命題2.15より
$Y$
は確率変数である.\ 

\section*{積分の基礎事項}
\begin{definition}[測度の積分]
  $(\Omega,\mathcal{F},\mu)$
  を測度空間とする.\ 
  $f:\Omega \to \mathbb{R}$
  を可測関数とすると,\ 
  $f$
  の
  $\Omega$
  上の積分を
  $\mu (f)$
  または,\ 
  $\int_{\Omega} f(\omega) \mu (d\omega)$
  と表記する.\ 

  $f = \sum_{i=1}^m a_i 1_{A_i},\ a_i \in \left[ 0, \infty\right],\ A_i \in \mathcal{F}$
  と有限和でかけるとき,\ 
  $f$
  を単純と呼び,\ 
  $f \in SF^+$
  と書く.\ 
  $f \in SF^+$
  のとき,\ 
  \begin{equation}
    \mu (f) := \sum_{i=1}^m a_i \mu(A_i)
  \end{equation}
  と定義し,\ 
  $f$
  が非負値関数である($f \in m\mathcal{F}^+$と書く)とき,\ 
  \begin{equation}
    \mu(f) := \sup\left\{ \mu(h) | h \in SF^+,\ h \leq f\right\}
  \end{equation}
  と定義する(教科書では
  $\lim$
  で定義されている).\ 
  また,\ 
  $f$
  が一般の可測関数である($f \in m\mathcal{F}$と書く)とき,\ 
  $f^+(\omega) := \max(f(\omega),0)$,\ 
  $f^-(\omega) := \max(-f(\omega),0)$
  とし,\ さらに
  $\mu(|f|)=\mu(f^+)+\mu(f^-) < \infty$
  であるとき,\ 
  \begin{equation}
    \mu(f) := \mu(f^+) - \mu(f^-)
  \end{equation}
  と定義する.\ 
\end{definition}

\begin{lemma}[単純な関数の積分の性質]\label{lemma:int_nature}
  $f,\ g \in SF^+$
  とする.\ 
  \begin{description}
    \item[(1)線型性]
      $\alpha,\ \beta \in \mathbb{R}$のとき,\ 
      $\mu(\alpha f + \beta g) = \alpha \mu(f) + \beta \mu(g)$
      となる.\ 
    \item[(2)単調性]
      $f \leq g$
      のとき,\ 
      $\mu(f) \leq \mu(g)$
      となる.\ 
  \end{description}
\end{lemma}

\begin{proof}
  $f = \sum_{i=1}^n a_i 1_{A_i}$, 
  $g = \sum_{j=1}^m b_j 1_{B_j}$
  とする.\ 
  $a_i,\ b_j \in \left[ 0, \infty\right]$,\ 
  $A_i,\ B_j$は各々異なる添字同士で互いに素である.\ 
  $C_{ij} = A_i \cap B_j$
  とすると,\ 
  $C_{ij}$
  も異なる
  $i,\ j$
  同士で互いに素である.\ 

  (1)
  \begin{align}
    \alpha f + \beta g
    & = \alpha \sum_{i=1}^n a_i 1_{A_i} + \beta \sum_{j=1}^m b_j 1_{B_j} \\
    & = \sum_{i,j} \left( \alpha a_i + \beta b_j \right) 1_{C_ij},
  \end{align}
  なので,\ 
  \begin{align}
    \mu(\alpha f + \beta g)
    & = \sum_{i,j} \left( \alpha a_i + \beta b_j\right) \mu(C_{ij}) \\
    & = \alpha \sum_{i=1}^n a_i \sum_{j=1}^m \mu(C_{ij}) + \beta \sum_{j=1}^m b_j \sum_{i=1}^n \mu(C_{ij}) \\
    & = \alpha \sum_{i=1}^n a_i \mu(A_i) + \beta \sum_{j=1}^m \mu(B_j) \\
    & = \alpha \mu(f) + \beta \mu(g).
  \end{align}

  (2)
  $I := \left\{(i,j) | C_{ij} \neq \emptyset \right\}$
  とすると,\ 
  \begin{align}
    g- f
    & = \sum_{i,j} (b_j - a_i) 1_{C_{ij}} \\
    & = \sum_{(i,j) \in I} (b_j - a_i) 1_{C_{ij}} \geq 0,
  \end{align}
  であるので,\ 
  $(i,j) \in I$
  ならば,\ 
  $b_j - a_i \geq 0$
  である.\ 
  よって,\ 
  $\mu(g-f) = \sum_{(i,j) \in I} (b_j-a_i) \mu(C_{ij}) \geq 0$
  であるので,\ 線形性より
  $\mu(f) \leq \mu(g)$.
\end{proof}

\section*{単調収束定理}
単調収束定理は測度論の中でも重要な定理で,\ 単純な関数で示せる性質を非負値関数,\ そして一般の可測関数に拡張するためや,\ Fatouの補題,\ Lebesugueの収束定理を示す際に使われる定理である.\ 
単調収束定理を証明するためにいくつかの命題,\ 補題を証明する.\ 

\begin{proposition}\label{prop:doubly}
  $\forall r, n \in \mathbb{N}$
  に対して,\ 
  $y_n^r \in \left[ 0,\infty \right]$
  であり,\ かつ次の2つを満たすとする.\ 
  \begin{itemize}
    \item 
      $y_n^r$
      が
      $r$
      を固定した時,\ 
      $n$
      について単調非減少で,\ 
      $y^r:=\lim_n y_n^r$
      が存在する.\ 
    \item
      $y_n^r$
      が
      $n$
      を固定した時,\ 
      $r$
      について単調非減少で,\ 
      $y_n:=\lim_r y_n^r$
      が存在する.\ 
  \end{itemize}
  このとき,\ 
  $y^\infty:=\lim_r y_n^r = \lim_n y_n^r=:y_\infty$
  となる.\ 
\end{proposition}

\begin{proof}
  任意の
  $\epsilon > 0$
  に対し,\ 
  $y_{n_0} > y_\infty - \frac{1}{2}\epsilon$
  となる
  $n_0 \in \mathbb{N}$
  が存在する.\ 
  また,\ 
  $y_{n_0}^{r_0} > y_{n_0} - \frac{1}{2}\epsilon$
  となる
  $r_0 \in \mathbb{N}$
  が存在する.\ 
  よって,\ 
  $y^{\infty} \geq y^{r_0} \geq y_{n_0}^{r_0} > y_{n_0} - \frac{1}{2} > y_{\infty} - \epsilon$
  となり,\ 
  $y^{\infty} \geq y_{\infty}$
  であることがわかる.\ 
  同様に,\ 
  $y_{\infty} \geq y^{\infty}$
  なので,\ 
  $y^{\infty} = y_{\infty}$
  が得られた.\ 
\end{proof}


\begin{lemma} \label{lemma:mono_set}
  $F_n \in \mathcal{F}$
  が単調非減少で
  $F=\bigcup_{n=1}^{\infty} F_n$
  とする.\ 
  このとき,\ 
  $\lim_{n} \mu(F_n) = \mu(F)$
  となる.\ 
\end{lemma}


\begin{proof}
  $G_1 = F_1$,\ 
  $G_n = F_n \setminus F_{n-1}\ (n\geq2)$
  とすると,\ 各
  $G_n$
  はそれぞれdisjointである.\ 
  よって,\ 
  $\mu(F_n) = \mu(\bigcup_{k=1}^n G_k) = \sum_{k=1}^n \mu(G_k)$
  であるので,\ 
  $\lim_{n} \mu(F_n) = \sum_{k=1}^{\infty} \mu(G_k) = \mu(F)$.
\end{proof}

\begin{lemma}
  $A \in \mathcal{F},\ h_n \in SF^+$,\ 
  $h_n$
  は各点において単調非減少で,\ 
  $1_A$
  に各点収束するとすると,\ 
  $\lim_n \mu(h_n) = \mu(A)$
  である.\ 
\end{lemma}

\begin{proof}
  補題\ref{lemma:int_nature}(2)より,\ 
  $\mu(h_n) \leq \mu(1_A) = \mu(A)$
  である.\ 
  よって,\ 
  $\liminf_{n} \mu(h_n) \geq \mu(A)$
  を示せば十分である.\ 

  $\epsilon > 0$
  とし,\ 
  $A_n := \left\{\omega \in A | h_n(\omega) > 1-\epsilon \right\}$
  とする.\ 
  $\omega \in A$
  であるとき,\ 
  $h_n$
  は
  $1_{A}$
  に各点収束するから,\ 
  $h_{n_0}(\omega) > 1 - \epsilon$
  となるような
  $n_0 \in \mathbb{N}$
  が存在するので,\ 
  $\omega \in \bigcup_{n=1}^{\infty} A_n$
  となり,\ 
  $A \subset \bigcup_{n=1}^{\infty} A_n$.\ 
  よって,\ 
  $A = \bigcup_{n=1}^{\infty} A_n$
  なので,\ 
  $A_n \nearrow A$
  である.\ 
  補題\ref{lemma:mono_set}より,\ 
  $\lim_n \mu(A_n) = \mu(A)$
  となる.\ 

  また,\ 
  $A_n$
  の定義から,\ 
  $(1-\epsilon)1_{A_n} \leq h_n$
  であるので,\ 補題\ref{lemma:int_nature}より,\ 
  $\mu((1-\epsilon)1_{A_n}) = (1-\epsilon)\mu(A_n) \leq \mu(h_n)$
  となり,\ 
  $\liminf_n \mu(h_n) \geq (1-\epsilon) \mu(A)$
  である.\ 
  $\epsilon>0$
  は任意にとってよいので
  $\liminf_n \mu(h_n) \geq \mu(A)$
  となる.\ 
\end{proof}

%\bibliographystyle{junsrt}
%\bibliography{}

\end{document}
