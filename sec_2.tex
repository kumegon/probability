\documentclass[a4j,11pt]{jarticle}
\usepackage{url}
\usepackage[dvipdfmx]{graphicx}
\usepackage{here}
\usepackage{amssymb}
\usepackage{amsmath,amssymb}
\usepackage[dvipdfmx]{hyperref}
\usepackage{enumerate}
\usepackage{mathtools}
\graphicspath{{picture/}}
\usepackage{ascmac}
\usepackage{bm}
\usepackage{amsthm}
\usepackage{algorithm}
\usepackage{algorithmic}
\usepackage{comment}

\setlength{\textwidth}{1.1\textwidth}
\setlength{\oddsidemargin}{-3pt}
\setlength{\evensidemargin}{\oddsidemargin}
\setlength{\topmargin}{-10mm}
\setlength{\headheight}{0mm}
\setlength{\headsep}{0mm}
\begin{comment}
\usepackage{listings,jlisting}
\lstset{language=c,
  basicstyle=\ttfamily\scriptsize,
  commentstyle=\textit,
  classoffset=1,
  keywordstyle=\emphseries,
  frame=tRBl,
  framesep=5pt,
  showstringspaces=false,
  numbers=left,
  stepnumber=1,
  numberstyle=\tiny,
  tabsize=2
}
\end{comment}

\renewcommand\thefootnote{\arabic{footnote})}

\theoremstyle{definition}
\renewcommand{\proofname}{\bf{証明}}
\newtheorem{theorem}{定理}
\newtheorem*{theorem*}{定理}
\newtheorem{definition}[theorem]{定義}
\newtheorem*{definition*}{定義}
\newtheorem{lemma}[theorem]{補題}
\newtheorem*{lemma*}{補題}
\newtheorem{corollary}[theorem]{系}
\newtheorem*{corollary*}{系}
\newtheorem{proposition}[theorem]{命題}
\newtheorem*{proposition*}{命題}
\newtheorem{nature}[theorem]{性質}
\newtheorem{remark}[theorem]{注意}

\newcommand{\argmax}{\mathop{\rm arg~max}\limits}
\newcommand{\argmin}{\mathop{\rm arg~min}\limits}
\newcommand{\DtoR}[2]{{#1 \rightarrow #2}}
\newcommand{\dtor}[2]{{#1 \mapsto #2}}
\newcommand{\prox}[1]{{{\rm prox}_{#1}}}
\newcommand{\inprod}[2]{{\langle #1,#2 \rangle}}

\makeatother
\setcounter{secnumdepth}{4}


\title{舟木確率論ゼミ}
\author{M1 久米啓太}
\date{\today}


\begin{document}
\maketitle


\section{命題2.25の証明の補足}
\subsection*{$F$が右連続であることの証明}
教科書では
$y_n:=x+\frac{1}{n}$
となる数列を用いて
$\lim_{n\to\infty}F(y_n) = F(x)$
を示しているが,\ 
$x$
に収束する任意の数列
$(z_n)_{n=1}^{\infty}\ (z_n>x)$
について
$\lim_{n\to\infty}F(z_n) = F(x)$
を示さないと,\ 
$F$
の右連続性を示すことができない.\ 
ここでは,\ もう少し詳細に
$F$
の右連続性を証明する.\ 

\begin{proof}
  $A_n:=\left( -\infty, x+\frac{1}{n}\right]$
  と定義すると,\ 
  $A_n \searrow A :=(-\infty, x]$
  となる.\ 
  よって,\ 測度の連続性より,\ 
  $\lim_{n\to\infty}\mu(A_n) = \mu(A)$
  となる.\ 
  つまり,\ これは
  \begin{equation}
    \forall \epsilon > 0 \ \exists N \in \mathbb{N} \ \ {\rm s.t.}\ \ n > N \Rightarrow |\mu(A_n) - \mu(A)| < \epsilon
  \end{equation}
  となることを意味している.\ 
  この
  $N$
  を用いて,\ 
  $\delta := \frac{1}{N+1}$
  とすると,\ 
  $0<y-x<\delta$
  のとき,\ 
  $y < x + \delta = x + \frac{1}{N+1}$
  なので,\ 
  $A_{N+1} \supset (-\infty,y]$
  となる.\ 
  よって,\ 
  \begin{align}
    |F(y)-F(x)|
    & = |\mu((-\infty,y]) - \mu(A)| \\
    & = \mu((-\infty,y]) - \mu(A) \\
    & \leq \mu(A_{N+1}) - \mu(A) \\
    & < \epsilon
  \end{align}
  となり,\ 
  $F$
  が右連続であることを示すことができた.\ 
\end{proof}
\subsection*{$Y$が確率変数であることの証明}
1)-3)を満たす
$\mathbb{R}$
上の関数
$F$
が与えられた時, (2.2)を満たすような
$(\mathbb{R},\mathcal{B}(\mathbb{R}))$
上の確率測度
$\mu$
が一意的に定まることを証明するために,\  
$Y(w):\left( 0,1\right)\to \mathbb{R}:w\mapsto \sup \left\{y \in \mathbb{R} | F(y) < w \right\}, \ w \in \left(0,1 \right)$
が確率変数であることを用いている.\ 
集合
$A$
に対する
$y^\star = \sup A$
は以下を満たす
$\mathbb{R}$
の元である.\ 
\begin{itemize}
  \item 
    任意の
    $y \in A$
    に対して,\ 
    $y \leq y^\star$.
  \item
    任意の
    $\epsilon > 0$
    に対して,\ ある
    $y \in A$
    が存在して,\ 
    $y^\star - \epsilon < y$.
\end{itemize}

教科書では
\begin{quote}
$Y$
は
$F$
の左連続な逆関数
\end{quote}
であると書かれているが,\ 
$F$
に逆関数が存在するとは限らないことに注意する.\ 
なぜならば,\ 1)-3)の条件だけでは
$F$
が全単射であることを保証できないからである.\ 

教科書では
$Y$
が左連続であることと,\ 左連続関数は可測関数となることを用いて証明しているが,\ 左連続関数が可測関数となる事実を用いずに証明できるので,\ その証明を紹介する.\ 

ちなみに
$Y$
が左連続であることの証明は以下の通りである.\ 
\begin{proof}
  $\forall w \in (0,1) \ \forall \epsilon >0 \ \exists \delta > 0  \ \  {\rm s.t.}\ \ 0 < w-a <\delta \Rightarrow |Y(w)-Y(a)| < \epsilon$
  を示せばよい.\ 
  $C(w) := \left\{y\in\mathbb{R}\ |\ F(y)<w \right\}$
  とすると,\ 
  $1 > w > a > 0$
  のとき,\ 
  $C(w) \supset C(a)$
  であるため,\ 
  $Y(w) \geq Y(a)$
  である.\ 
  $Y$
  の定義より,\ 任意の
  $\epsilon > 0$
  に対し,\ ある
  $y \in C(w)$
  が存在して,\ 
  $Y(w) - \epsilon < y$
  とできる.\ 
  $F(y) < w$
  なので,\ ある
  $\delta > 0$
  が存在して,\ 
  $F(y) < w - \delta$
  とでき,\ 
  $y \in C(w-\delta)$,\ 
  つまり
  $y \leq Y(w-\delta)$
  となる.\ 
  この
  $\delta$
  を用いて,\ 
  $0 < w-a<\delta$
  を満たす任意の
  $a$
  を考える.\ 
  $w-\delta < a$
  なので,\ 
  $Y(w-\delta) \leq Y(a)$
  となることに注意すると,\ 
  \begin{align}
    |Y(w)-Y(a)|
    & = Y(w) - Y(a) \\
    & \leq Y(w) - Y(w-\delta) \\
    & \leq Y(w) - y \\
    & < \epsilon
  \end{align}
  が成立する.\ 
\end{proof}

以下,\ 
$Y$
が確率変数であることを証明する.\ 
任意の
$x\in \mathbb{R}$
に対し,\ 
$A(x):=\left\{ w \in \left( 0,1\right) | Y(w) \leq x \right\} = \left\{ w\in \left( 0,1\right) | w\leq F(x) \right\}=:B(x)$
が成り立つことを示す.\ 

まず,\ 
$A(x) \subset B(x)$
を示す.\ 
$w \in A(x)$
とする.\ 
ここで,\ 
$w > F(x)$
であると仮定する.\ 
$F$
の右連続性から,\ 
$\epsilon := w - F(x)$
とすると、ある
$\delta > 0$
が存在して、
$0 < y - x < \delta \Rightarrow |F(y)-F(x)| < \epsilon$
とできる。
$0 < y - x < \delta$
を満たす
$y$
を用いると、
$|F(y)-F(x)| = F(y)-F(x) < \epsilon$
より、
$F(y) < F(x)+\epsilon = w$
となる。
よって,\ 
$Y(w) \geq y > x$
となる.\ 
これは
$w \in A(x)$
,\ つまり
$Y(w) \leq x$
であることに矛盾するので,\ 
$w \leq F(x)$
である.\ 
以上より,\ 
$w \in B(x)$,\ 
$A(x) \subset B(x)$.\ 

次に,\ 
$A(x) \supset B(x)$
を示す.\ 
$w \in B(x)$
とすると,\ 
$w \leq F(x)$
である.\ 
$x \not\in \left\{ y \in \mathbb{R} | F(y) < w\right\}$
より,\ 
$Y(w) \leq x$
である.\ 
以上より,\ 
$w \in A(x),\ $
$A(x) \supset B(x)$.\ 

\begin{equation}
  B(x) = \left\{ w  \in\left( 0,1\right) | w \leq F(x)\right\} = 
  \begin{cases}
    \left( 0, F(x)\right] & {\rm if } \ F(x) > 0 \\
    \emptyset & {\rm otherwise}
  \end{cases}
\end{equation}
なので,\ 明らかに
$B(x) \in \mathcal{B}((0,1))$
である.\ 
よって,\ 任意の
$x\in \mathbb{R}$
において
$A(x)$
も可測集合であるので,\ 命題2.15より
$Y$
は確率変数である.\ 


\section{特異型の分布関数の例:Cantor関数}
Cantor関数は特異型の分布関数の
$1$
つで,\ 測度が
$0$
の集合にすべての変動が集中しており,\ 悪魔の階段関数と言われる.\ 

\begin{definition}[Cantor関数の再帰的定義]
  関数
  $F_0,F_1,F_2,\ldots:[0,1]\to[0,1]$
  を再帰的に次の通りに定義する.\ 
  \begin{align}
    F_0(x)
    & := x\\
    F_n(x)
    & :=
    \begin{cases}
      \frac{1}{2}F_{n-1}(3x) & \left(0 \leq x \leq \frac{1}{3}\right) \\
      \frac{1}{2} & \left( \frac{1}{3} < x < \frac{2}{3}\right) \\
      \frac{1}{2} + \frac{1}{2}F_{n-1}(3x-2) & \left( \frac{2}{3} \leq x \leq 1\right)
    \end{cases}
  \end{align}
  $F_n$
  はある関数
  $F:[0,1]\to[0,1]$
  に一様収束し,\ この関数をカントール関数という.\ 
\end{definition}

構成から,\ 各
$F_n$
は連続な単調関数で
$F_n(0) = 0$,
$F_n(1) = 1$
であることがわかる.\ 

$F_n$
がある関数
$F$
に一様収束することを示す.\ 
任意の
$n \in \mathbb{N},\ x \in [0,1]$
に対し,\ 
\begin{align}
  |F_{n+1}(x) - F_{n}(x)|
  & \leq \max_{y\in[0,1]} |F_{n+1}(y)-F_{n}(y)| \\
  & = \max\left( \max_{y\in [0,1/3]} \left| \frac{1}{2} F_n(3y) - \frac{1}{2} F_{n-1}(3y)\right| , 0, \max_{y \in [2/3,1]} \left| \frac{1}{2} F_n(3y-2) - \frac{1}{2}F_{n-1}(3y-2) \right| \right) \\
  & = \frac{1}{2}\max_{y\in[0,1]}|F_n(y) - F_{n+1}(y)| \\
  & = 2^{-n} \max_{y \in [0,1]} |F_1(y) - F_0(y)| \\
  & \leq 2^{-n}
\end{align}
となる.\ 
$m > n \geq 1,\ x \in [0,1]$
に対し,\ 
\begin{align}
  |F_m(x) - F_n(x)|
  & = \left| \sum_{k=n}^{m-1}(F_{k+1}(x) - F_k(x)) \right| \\
  & \leq \sum_{k=n}^{m-1} |F_{k+1}(x) - F_k(x)| \\
  & \leq \sum_{k=n}^{m=1} 2^{-k} \\
  & = \sum_{k=0}^{m-1} 2^{-k} - \sum_{k=0}^{n-1} 2^{-k} \\
  & = 2^{-(n-1)} - 2^{-(m-1)} \\
  & \xrightarrow{n,m\to \infty} 0
\end{align}
となる.\ 
よって,\ 
$F_n(x)$
はコーシー列で,\ 実数の完備性より
$F_n$
はある関数
$F$
に各点収束することがわかる.\ 
また,\ 任意の
$n \in \mathbb{N},\ x \in [0,1]$
に対し,\ 
\begin{align}
  |F(x) - F_n(x)|
  & = \left| \sum_{k=n}^{\infty} F_{k+1}(x) - F_{k}(x) \right| \\
  & \leq \sum_{k=n}^{\infty} |F_{k+1}(x) - F_k(x) | \\
  & \leq \sum_{k=n}^{\infty} 2^{-k} \\
  & = 2^{-(n-1)}
\end{align}
である.\ 
任意の
$\epsilon > 0$
に対し,\ 
$N = \max\left(\lfloor -\log_2 \epsilon + 1 \rfloor, 1 \right)$
とすれば,\ 
$n > N$
であるとき,\ 任意の
$x \in [0,1]$
で
$|F(x)-F_n(x)| < \epsilon$
が成り立ち,\ 
$F_n$
が
$F$
に一様収束することがわかる.\ 

\begin{remark}
  連続な関数列が一様収束するならば収束先の関数は連続となるので,\ カントール関数は連続関数である.\ 
\end{remark}

次にCantor関数が特異型であることを示す.\ 
任意の
$k \in \mathbb{N}$
に対し,\ 集合
$A_k$
を
\begin{equation}
  A_k :=\left( \sum_{i=1}^{n-1}\frac{\alpha_i}{3^i} + \frac{1}{3^n}, \sum_{i=1}^{n-1}\frac{\alpha_i}{3^i} + \frac{2}{3^n}\right)
\end{equation}
と定義する.\ 
ただし,\ 各
$\alpha_i$
は
$k = \sum_{i=1}^{n-1} 2^{i-1}\beta_i$
と2進数展開したときの
$\beta_i$
を用いて,\ 
\begin{equation}
  \alpha_i := 
  \begin{cases}
    0 & (\beta_i = 0) \\
    2 & (\beta_i = 1)
  \end{cases}
\end{equation}
と定義される.\ 
各
$A_k$
は開区間であるため,\ 
$A_k \in \mathcal{B}(\mathbb{R})$
である.\ 

$F$
の構成から各
$A_k$
において
$F$
の値は変化しない.\ 
よって,\ 
\begin{align}
  dF(A_k)
  & = \mu(A_k) \\
  & = F\left( \sum_{i=1}^{n-1}\frac{\alpha_i}{3^i} + \frac{2}{3^n}\right) - F\left( \sum_{i=1}^{n-1}\frac{\alpha_i}{3^i} + \frac{1}{3^n}\right) \\
  & = 0
\end{align}
である.\ 
$\mu$
の
$\sigma$-加法性より,\ 
\begin{align}
  \mu\left( \bigcup_{l=1}^{\infty} A_l\right)
  & = \sum_{l=1}^{\infty} \mu(A_l) \\
  & = 0
\end{align}
となる.\ 

$m$
をルベーグ測度とすると,\ 
$m(A_k) = \frac{1}{3^n}$
となる.\ 
また,\ 
$l \in \mathbb{N}$
において,\ 
$m(A_l) = \frac{1}{3^n}$
となる
$l$
の個数は
$2^{n-1}$
個である.\ 
ルベーグ測度の
$\sigma$-加法性より,\ 
\begin{align}
  m\left( \bigcup_{l=1}^{\infty} A_l\right)
  & = \sum_{l=1}^{\infty} m(A_l) \\
  & = \sum_{n=1}^{\infty} \frac{2^{n-1}}{3^n} \\
  & = \frac{1}{2} \sum_{n=1}^{\infty} \left( \frac{2}{3}\right)^n \\
  & = 1
\end{align}
となる.\ 

$B := [0,1] \cap \left( \bigcup_{l=1}^{\infty}A_l\right)^c \in \mathcal{B}(\mathbb{R})$
と定義する.\ 
\begin{align}
  dF(B) & = 1 - dF\left( \bigcup_{l=1}^{\infty}A_l\right) = 1 \\
  m(B) & = 1 - m\left( \bigcup_{l=1}^{\infty}A_l\right) = 0
\end{align}
となるので,\ Cantor関数は特異型であることがわかった.\ 


\section{H\"olderの不等式の証明}
(1)
$p,\ q$
が仮定を満たすとする. 

まず,  任意の非負数
$a,\ b$
に対し, 
\begin{equation}
  g(a):=\frac{1}{p}a^p + \frac{1}{q}b^q - ab\geq 0
\end{equation}
が成り立つことを示す.\ 
$g'(a) = a^{p-1} - b,\ g''(a)=(p-1)a^{p-2} \geq 0$
なので,\ 
$g'(a) = 0 \Leftrightarrow a = b^{\frac{1}{p-1}}$
のとき,\ 
$g(a)$
は最小となる.\ 
したがって,\ 
\begin{align}
  g(a)
  & \geq g(b^{\frac{1}{p-1}}) \\
  & = \frac{1}{p}b^{\frac{p}{p-1}} + \frac{1}{q}b^q - b^{\frac{1}{p-1}}b \\
  & = \frac{1}{p}b^{q} + \frac{1}{q}b^q + b^q \\
  & \geq 0
\end{align}
となる.\ 
下から2番目の式は
$\frac{1}{p}+\frac{1}{q}=1$
から
$q=\frac{p}{p-1}$
となることを用いた.\ 

今,\ 
$\omega \in \Omega$
に対し,\ 
$a = \frac{|X(\omega)|}{\mathbb{E}[|X|^p]^{\frac{1}{p}}}$,
$b = \frac{|Y(\omega)|}{\mathbb{E}[|Y|^q]^{\frac{1}{q}}}$
とすると,\ 共に非負であるので,\ 
\begin{align}
  \frac{1}{p}\left( \frac{|X(\omega)|}{\mathbb{E}[|X|^p]^{\frac{1}{p}}}\right)^p + \frac{1}{q}\left( \frac{|Y(\omega)|}{\mathbb{E}[|Y|^q]^{\frac{1}{q}}}\right)^q
  & = \frac{1}{p} \frac{|X(\omega)|^p}{\mathbb{E}[|X|]^p} + \frac{1}{q} \frac{|Y(\omega)|^q}{\mathbb{E}[|Y|^q]} \\
  & \geq \frac{|X(\omega)||Y(\omega)|}{\mathbb{E}(|X|^p)^{\frac{1}{p}}\mathbb{E}(|Y|^q)^{\frac{1}{q}}}
\end{align}
が成り立つ.\ 
よって,\ 積分の単調性と線形性より,\ 
\begin{align}
  \frac{1}{p}\frac{\mathbb{E}[|X|^p]}{\mathbb{E}[|X|^p]} + \frac{1}{q}\frac{\mathbb{E}(|Y|^q)}{\mathbb{E}(|Y|^q)}
  & = \frac{1}{p} + \frac{1}{q} \\
  & = 1 \\
  & \geq \frac{\mathbb{E}[|XY|]}{\mathbb{E}(|X|^p)^{\frac{1}{p}}\mathbb{E}(|Y|^q)^{\frac{1}{q}}} \\
  & \geq \frac{|\mathbb{E}[XY]|}{\mathbb{E}(|X|^p)^{\frac{1}{p}}\mathbb{E}(|Y|^q)^{\frac{1}{q}}}
\end{align}
となり,\ 
$|\mathbb{E}[XY]| \leq \mathbb{E}(|X|^p)^{\frac{1}{p}}\mathbb{E}(|Y|^q)^{\frac{1}{q}}$
であることがわかった.\ 

(2)
$A:= \left\{\omega \in \Omega \ |\  |X(\omega)| > \|X\|_\infty \right\}$
とすると
$\|X\|_\infty$
の定義より,\ 
$P(A) = 0$
となる.\ 
また,\ 
$\omega \in \Omega \setminus A$
において
$|X(\omega)| \leq \|X\|_\infty$
である.\ 
よって,\ 
\begin{align}
  |\mathbb{E}[XY]|
  & \leq \mathbb{E}[|XY|] \\
  & = \int_{\Omega} |X(\omega)| |Y(\omega)| P(d\omega) \\
  & = \int_{\Omega} |X(\omega)| |Y(\omega)| (1_{A} + 1_{\Omega \setminus A}) P(d\omega) \\
  & = \int_{\Omega} |X(\omega)| |Y(\omega)| 1_{A} P(d\omega) + \int_{\Omega} |X(\omega)| |Y(\omega)| 1_{\Omega \setminus A} P(d\omega) \\
  & = \int_{A} |X(\omega)| |Y(\omega)| P(d\omega) + \int_{\Omega \setminus A} |X(\omega)| |Y(\omega)| P(d\omega) \\
  & = \int_{\Omega \setminus A} |X(\omega)| |Y(\omega)| P(d\omega) \\
  & \leq \|X\|_\infty \int_{\Omega \setminus A} |Y(\omega)| P(d\omega) \\
  & \leq \|X\|_\infty \int_{\Omega} |Y(\omega)| P(d\omega) \\
  & = \|X\|_\infty \mathbb{E}[|Y|]
\end{align}
となり,\ 
$|\mathbb{E}[XY]| \leq \|X\|_\infty \mathbb{E}[|Y|]$
であることがわかった.\ 

\section{確率変数列の収束}
\begin{comment} %法則収束
\begin{lemma}
  確率変数列列
  $(X_n)_{n=1}^{\infty}$
  が
  確率変数
  $X$
  に法則収束することと,\ 分布関数列
  $(F_{X_n})_{n=1}^{\infty}$
  が
  $F_X$
  に各点収束することは同値である.\ 
\end{lemma}
\begin{proof}
  (1)必要性

  $(X_n)_{n=1}^{\infty}$
  が
  $X$
  に法則収束すると仮定する.\ 
  $\mathbb{E}[f(X)] = \int_{\mathbb{R}}f(x) P_X(dx)$
  となるので,\ 
  $\lim_{n\to\infty} \int_{\mathbb{R}} f(x) P_{X_n}(dx) = \int_{\mathbb{R}} f(x) P_X(dx)$
  となる.\ 
  $A \in \mathcal{O}(\mathbb{R})$
  とすると,\ 
  $g_k \in C_b(\mathbb{R}) \nearrow 1_A$
  となる関数列
  $(g_k)_{k=1}^{\infty}$
  が存在する.\ 
  $\int_{\mathbb{R}} 1_A(x) P_{X_n}(dx) = P_{X_n}(A) \geq \int_{\mathbb{R}} g_k(x) P_{X_n}(dx)$
  なので,\ 仮定より
  $\liminf_{n\to\infty} P_{X_n}(dx) \geq \liminf_{n\to\infty} \int_{\mathbb{R}} g_k(x) P_{X_n}(dx) = \int_{\mathbb{R}} g_k(x) P_X(dx)$
  が成り立つ.\ 
  単調収束定理より
  $\lim_{k\to\infty} \int_{\mathbb{R}} g_k(x) P_X(dx) = \int_\mathbb{R} 1_A(x)P_X(dx) = P(A)$
  が成り立つので,\ 
  \begin{equation} \label{eq:inf}
    \liminf_{n\to\infty} P_{X_n}(A) \geq P(A)
  \end{equation}
  となる.\ 
  また,\ 
  $B = A^c$
  とすると,\ 
  $\limsup_{n\to\infty} P_{X_n}(B) = \limsup_{n\to\infty} 1-P_{X_n}(A) = 1 - \liminf_{n\to\infty}P_{X_n}(A) \leq 1 - P_X(A) = P_X(B)$
  が成り立つ.\ 
  よって,\ 任意の
  $B \in \mathcal{O}(\mathbb{R})^c$
  に対し,\ 
  \begin{equation} \label{eq:inf}
    \limsup_{n\to\infty} P_{X_n}(B) \leq P_X(B)
  \end{equation}
  となることがわかる.\ 
  以上より,\ 任意の
  $t \in \mathbb{R}$
  に対し,\ 
  \begin{align}
    \limsup_{n\to\infty} F_{X_n}(t) \\
    & = \limsup_{n\to\infty} P_{X_n}((-\infty, t]) 
    & \leq P_X((-\infty, t]) \\
    & = F_X(t) \\
    & = P_X
  \end{align}%ここ書いてない
  $\lim_{n\to\infty} \int_{\mathbb{R}} 1_A(x) P_{X_n}(dx) = \lim_{n\to\infty} P_{X_n}(A)$,
  $\int_{\mathbb{R}} 1_A(x) P_X(dx) = P_X(A)$
  となる.\ 
  よって,\ 
  $\lim_{n\to\infty} P_{X_n}(A) = P_X(A)$
  であるので,\ 
  $\lim_{n\to\infty} F_{X_n} = F_X$
  とわかる.\ 

  (2)十分性

  $\lim_{n\to\infty}F_{X_n} = F_X$
  と仮定する.\ 
  $\pi$-$\lambda$定理より,\ 
  $\lim_{n\to\infty}P_{X_n} = P_X$
  である.\ 
  $A \in \mathcal{B}(\mathbb{R})$
  とする.\ 
  $\lim_{n\to\infty} \int_\mathbb{R} 1_A(x) P_{X_n}(dx) = \lim_{n\to\infty} P_{X_n}(A)$,
  $\int_{\mathbb{R}} 1_A(x) P_X(dx) = P_X(A)$
  より,\ 
  $\lim_{n\to\infty} \int_{\mathbb{R}} 1_A(x) P_{X_n}(dx) = \int_{\mathbb{R}} 1_A(x) P_X(dx)$
  とわかる.\ 
\end{proof}

\end{comment}

\subsection{p次平均収束して概収束しない例}
例2.48の確率変数列
$X_{n,k}(\omega) = 1_{\left( \frac{k-1}{n},\frac{k}{n}\right)}(\omega)$
を考える.\ 
$1\leq k \leq n$
でなので,\ 与えられた
$n,k$
に対し,\ 
$m = \frac{1}{2}n(n-1)+k$
とすると,\ 
$m$
と
$(n,k)$
は1対1に対応できる.\ 
新しい確率変数列
$(Y_m)_{m=1}^{\infty}$
を
$m$
に対応する
$(n,k)$
を用いて
$Y_m:= X_{n,k}$
と定義する.\ 
確率変数列
$(Y_m)_{m=1}^\infty$
が確率変数
$X(\omega) = 0 (\forall \omega \in \Omega)$
にp次平均収束することを示す.\ 
\begin{align}
  \mathbb{E}[|Y_m-X|^p]
  & = \mathbb{E}[|Y_m|^p] \\
  & = \int_{\Omega} |Y_m|^p P(d\omega) \\
  & = \int_{\Omega} |1_{\left( \frac{k-1}{n}, \frac{k}{n}\right)}(\omega)|^p P(d\omega) \\
  & = \int_{\Omega} 1_{\left( \frac{k-1}{n}, \frac{k}{n}\right)}(\omega) P(d\omega)\\
  & = P\left( \left( \frac{k-1}{n}, \frac{k}{n}\right)\right) \\
  & = \frac{1}{n}
\end{align}
$m\to \infty$
とすると,\ 
$n\to \infty$
となる.\ 
よって,\ 
$\lim_{m\to \infty} \mathbb{E}[|Y_m-X|^p] = \lim_{n\to \infty}\frac{1}{n} = 0$
となり,\ 
$(Y_m)_{m=1}^{\infty}$
が
$X$
にp次平均収束することがわかった.\ 

次に
$(Y_m)_{m=1}^{\infty}$
が
$X$
に概収束しないことを示す.\ 
まず,\ 
$(Y_m(\omega))_{m=1}^{\infty}$
が任意の
$\omega \in \Omega$
において
$X(\omega)$
に収束しないことを示す.\ 
任意の
$\omega \in \Omega$,\ 
$M_0 \in \mathbb{N}$
に対し,\ ある
$m > M_0$
が存在し,\ 
$\omega \in \left( \frac{k-1}{n}, \frac{k}{n}\right)$
とできる.\ 
ただし,\ 
$(n,k)$
は
$m$
に対応する自然数である.\ 
$|Y_m(\omega) - X(\omega)| = |1_{\left( \frac{k-1}{n}, \frac{k}{n}\right)}(\omega)| = 1$
となり,\ 
$(Y_m(\omega))_{m=1}^{\infty}$
が
$X(\omega)$
に収束しないことがわかった.\ 
したがって,\ 
$A:= \left\{\omega \in \Omega \ | \ \lim_{m\to \infty}Y_m(\omega) = X(\omega) = 0 \right\} = \Omega$,\ 
つまり
$P(\lim_{m\to\infty}Y_m = X) = P(\emptyset) = 0$
となり,\ 
$(Y_m)_{m=1}^{\infty}$
が
$X$
に概収束しないことがわかった.\ 


\subsection{概収束してp次平均収束しない例}
例2.48の確率変数列
$X_n(\omega)=n1_{(0,\frac{1}{n})}(\omega)$
を考える.\ 
$(X_n)_{n=1}^\infty$
が確率変数
$X(\omega)(\omega)=0(\forall \omega \in \Omega)$
に概収束することを示す.\ 
$(X_n)_{n=1}^{\infty}$
が
$X$
に各点収束することを示せば十分である.\ 
なぜならば,\ 各点収束するならば
$\left\{\omega \in \Omega \ | \ \lim_{n\to\infty}X_n(\omega) = X(\omega) \right\} = \Omega$
となり,\ 
$P(\Omega) = 1$
となるからである.\ 

任意の
$\omega \in \Omega$
に対し,\ 
$N_0 := \lfloor \frac{1}{\omega} \rfloor$
とする.\ 
$n > N_0$
のとき,\ 
$|X_n(\omega)-X(\omega)| = |n1_{(0,\frac{1}{n})(\omega)}| = 0$
となる.\ 
よって,\ 
$(X_n)_{n=1}^{\infty}$
が
$X$
に各点収束し,\ 概収束することがわかった.\ 

次に
$(X_n)_{n=1}^{\infty}$
が
$X$
にp次平均収束しないことを示す.\ 
\begin{align}
  \mathbb{E}[|X_n-X|^p]
  & = \int_{\Omega} |X_n(\omega) - X(\omega)|^p P(d\omega) \\
  & = \int_{\Omega} |n1_{\left( 0,\frac{1}{n}\right)}(\omega)|^p P(d\omega) \\
  & = n^p \int_{\Omega} 1_{\left( 0, \frac{1}{n}\right)}(\omega) P(d\omega) \\
  & = n^p P\left( \left( 0,\frac{1}{n}\right)\right) \\
  & = n^p \frac{1}{n} \\
  & = n^{p-1}
\end{align}
であるので,\ 
$\lim_{n\to\infty} \mathbb{E}[|X_n-X|^p] = \lim_{n\to\infty} n^{p-1} > 0$
となる.\ 


\section{ルベーグ積分}
\subsection{表記法}
  $(\Omega,\mathcal{F},\mu)$
  を測度空間とし,\ 
  $f:\Omega \to \mathbb{R}$
  を可測関数とする.\ 

  $f = \sum_{i=1}^m a_i 1_{A_i},\ a_i \in \left[ 0, \infty\right],\ A_i \in \mathcal{F}$
  と有限和でかけるとき,\ 
  $f$
  を単関数と呼び,\ 
  $f \in SF^+$
  と書く.\ 
  $f$
  が非負値関数であるとき,\ 
  $f \in m\mathcal{F}^+$
  と書く.\ 
  $f$
  が一般の可測関数であるとき,\ 
  $f \in m\mathcal{F}$
  と書く.\ 

  集合列
  $A_n$
  が単調増加し,\ 
  $\bigcup_{n=1}^{\infty} A_n = A$
  となるとき,\ 
  $A_n \nearrow A$
  と書く.\ 

  また,\ 関数列
  $f_n$
  が各点で単調増加
  $(f_n(\omega) \leq f_{n+1}(\omega) \ldots)$
  し,\ 
  $\lim_{n} f_n(\omega) = f(\omega)$
  となるとき,\ 
  $f_n \nearrow f$
  と書く.\ 

\subsection{積分の基礎事項}
\begin{definition}[測度の積分]
  $(\Omega,\mathcal{F},\mu)$
  を測度空間とする.\ 
  $f:\Omega \to \mathbb{R}$
  を可測関数とすると,\ 
  $f$
  の
  $\Omega$
  上の積分を
  $\mu (f)$
  または,\ 
  $\int_{\Omega} f(\omega) \mu (d\omega)$
  と表記する.\ 

  $f \in SF^+$
  のとき,\ 
  \begin{equation}
    \mu (f) := \sum_{i=1}^m a_i \mu(A_i)
  \end{equation}
  と定義しする.\  
  $f \in m\mathcal{F}^+$
  のとき,\ 
  \begin{equation}
    \mu(f) := \sup\left\{ \mu(h) | h \in SF^+,\ h \leq f\right\}
  \end{equation}
  と定義する(教科書とは異なることに注意).\ 
  また,\ 
  $f\in m\mathcal{F}$
  であるとき,\ 
  $f^+(\omega) := \max(f(\omega),0)$,\ 
  $f^-(\omega) := \max(-f(\omega),0)$
  とし,\ さらに
  $\mu(f^+)\, \mu(f^-)$
  のうち少なくとも片方が有限な値を取るとき,\ 
  \begin{equation}
    \mu(f) := \mu(f^+) - \mu(f^-)
  \end{equation}
  と定義する.\ 
\end{definition}

\begin{lemma}[単関数の積分の性質]\label{lemma:int_nature}
  $f,\ g \in SF^+$
  とする.\ 
  \begin{description}
    \item[(1)線型性]
      $\alpha,\ \beta \in \mathbb{R}$のとき,\ 
      $\mu(\alpha f + \beta g) = \alpha \mu(f) + \beta \mu(g)$
      となる.\ 
    \item[(2)単調性]
      $f \leq g$
      のとき,\ 
      $\mu(f) \leq \mu(g)$
      となる.\ 
  \end{description}
\end{lemma}

\begin{proof}
  $f = \sum_{i=1}^n a_i 1_{A_i}$, 
  $g = \sum_{j=1}^m b_j 1_{B_j}$
  とする.\ 
  $a_i,\ b_j \in \left[ 0, \infty\right]$,\ 
  $A_i,\ B_j$は各々異なる添字同士で互いに素である.\ 
  $C_{ij} = A_i \cap B_j$
  とすると,\ 
  $C_{ij}$
  も異なる
  $i,\ j$
  同士で互いに素である.\ 

  (1)
  \begin{align}
    \alpha f + \beta g
    & = \alpha \sum_{i=1}^n a_i 1_{A_i} + \beta \sum_{j=1}^m b_j 1_{B_j} \\
    & = \sum_{i,j} \left( \alpha a_i + \beta b_j \right) 1_{C_ij},
  \end{align}
  なので,\ 
  \begin{align}
    \mu(\alpha f + \beta g)
    & = \sum_{i,j} \left( \alpha a_i + \beta b_j\right) \mu(C_{ij}) \\
    & = \alpha \sum_{i=1}^n a_i \sum_{j=1}^m \mu(C_{ij}) + \beta \sum_{j=1}^m b_j \sum_{i=1}^n \mu(C_{ij}) \\
    & = \alpha \sum_{i=1}^n a_i \mu(A_i) + \beta \sum_{j=1}^m \mu(B_j) \\
    & = \alpha \mu(f) + \beta \mu(g).
  \end{align}

  (2)
  $I := \left\{(i,j) | C_{ij} \neq \emptyset \right\}$
  とすると,\ 
  \begin{align}
    g- f
    & = \sum_{i,j} (b_j - a_i) 1_{C_{ij}} \\
    & = \sum_{(i,j) \in I} (b_j - a_i) 1_{C_{ij}} \geq 0,
  \end{align}
  であるので,\ 
  $(i,j) \in I$
  ならば,\ 
  $b_j - a_i \geq 0$
  である.\ 
  よって,\ 
  $\mu(g-f) = \sum_{(i,j) \in I} (b_j-a_i) \mu(C_{ij}) \geq 0$
  であるので,\ 線形性より
  $\mu(f) \leq \mu(g)$.
\end{proof}

\subsection{単調収束定理}
単調収束定理は測度論の中でも重要な定理で,\ 単関数で示せる性質を非負値関数,\ そして一般の可測関数に拡張するためや,\ Fatouの補題,\ Lebesugueの収束定理を示す際に使われる定理である.\ 
単調収束定理を証明するためにいくつかの命題,\ 補題を証明する.\ 

\begin{proposition}\label{prop:doubly}
  $\forall r, n \in \mathbb{N}$
  に対して,\ 
  $y_n^r \in \left[ 0,\infty \right]$
  であり,\ かつ次の2つを満たすとする.\ 
  \begin{itemize}
    \item 
      $y_n^r$
      が
      $r$
      を固定した時,\ 
      $n$
      について単調増加で,\ 
      $y^r:=\lim_n y_n^r$
      が存在する.\ 
    \item
      $y_n^r$
      が
      $n$
      を固定した時,\ 
      $r$
      について単調増加で,\ 
      $y_n:=\lim_r y_n^r$
      が存在する.\ 
  \end{itemize}
  このとき,\ 
  $y^\infty:=\lim_r y^r = \lim_n y_n=:y_\infty$
  となる.\ 
\end{proposition}

\begin{proof}
  任意の
  $\epsilon > 0$
  に対し,\ 
  $y_{n_0} > y_\infty - \frac{1}{2}\epsilon$
  となる
  $n_0 \in \mathbb{N}$
  が存在する.\ 
  また,\ 
  $y_{n_0}^{r_0} > y_{n_0} - \frac{1}{2}\epsilon$
  となる
  $r_0 \in \mathbb{N}$
  が存在する.\ 
  よって,\ 
  $y^{\infty} \geq y^{r_0} \geq y_{n_0}^{r_0} > y_{n_0} - \frac{1}{2} > y_{\infty} - \epsilon$
  となり,\ 
  $y^{\infty} \geq y_{\infty}$
  であることがわかる.\ 
  同様に,\ 
  $y_{\infty} \geq y^{\infty}$
  なので,\ 
  $y^{\infty} = y_{\infty}$
  が得られた.\ 
\end{proof}


\begin{lemma} \label{lemma:mono_set}
  $F_n \in \mathcal{F}$
  が
  $F_n \nearrow F$
  とする.\ 
  このとき,\ 
  $\lim_{n} \mu(F_n) = \mu(F)$
  となる.\ 
\end{lemma}


\begin{proof}
  $G_1 = F_1$,\ 
  $G_n = F_n \setminus F_{n-1}\ (n\geq2)$
  とすると,\ 各
  $G_n$
  はそれぞれ互いに素である.\ 
  よって,\ 
  $\mu(F_n) = \mu(\bigcup_{k=1}^n G_k) = \sum_{k=1}^n \mu(G_k)$
  であるので,\ 
  $\lim_{n} \mu(F_n) = \lim_n \mu\left( \bigcup_{k=1}^n G_k\right) = \lim_n \sum_{k=1}^n \mu(G_k)  = \sum_{k=1}^{\infty} \mu(G_k) = \mu\left( \bigcup_{k=1}^{\infty} G_k\right) = \mu\left( \bigcup_{n=1}^{\infty}F_n\right) = \mu(F)$.
\end{proof}

\begin{lemma}\label{lemma:point_convergence}
  (1)
  $A \in \mathcal{F},\ h_n \in SF^+$,\ 
  $h_n \nearrow 1_A$
  とする.\ 
  このとき,\ 
  $\lim_n \mu(h_n) = \mu(A)$
  である.\ 

  (2)
  $f,\ g_n \in SF^+$
  で,\ 
  $g_n \nearrow f$
  とする.\ 
  このとき,\ 
  $\lim_n \mu(g_n) = \mu(f)$
  である.\ 
\end{lemma}

\begin{proof}
  (1)
  補題\ref{lemma:int_nature}(2)より,\ 
  $\mu(h_n) \leq \mu(1_A) = \mu(A)$
  である.\ 
  よって,\ 
  $\liminf_{n} \mu(h_n) \geq \mu(A)$
  を示せば十分である.\ 

  $\epsilon > 0$
  とし,\ 
  $A_n := \left\{\omega \in A | h_n(\omega) > 1-\epsilon \right\}$
  とする.\ 
  $\omega \in A$
  であるとき,\ 
  $h_n$
  は
  $1_{A}$
  に各点収束するから,\ 
  $h_{n_0}(\omega) > 1 - \epsilon$
  となるような
  $n_0 \in \mathbb{N}$
  が存在するので,\ 
  $\omega \in \bigcup_{n=1}^{\infty} A_n$
  となり,\ 
  $A \subset \bigcup_{n=1}^{\infty} A_n$.\ 
  よって,\ 
  $A = \bigcup_{n=1}^{\infty} A_n$
  なので,\ 
  $A_n \nearrow A$
  である.\ 
  補題\ref{lemma:mono_set}より,\ 
  $\lim_n \mu(A_n) = \mu(A)$
  となる.\ 

  また,\ 
  $A_n$
  の定義から,\ 
  $(1-\epsilon)1_{A_n} \leq h_n$
  であるので,\ 補題\ref{lemma:int_nature}より,\ 
  $\mu((1-\epsilon)1_{A_n}) = (1-\epsilon)\mu(A_n) \leq \mu(h_n)$
  となり,\ 
  $\liminf_n \mu(h_n) \geq (1-\epsilon) \mu(A)$
  である.\ 
  $\epsilon>0$
  は任意にとってよいので
  $\liminf_n \mu(h_n) \geq \mu(A)$
  となる.\ 

  (2)
  $f = \sum_{i=1}^m a_i 1_{A_i}$
  とし,\ 
  各
  $A_i$
  は互いに素で,\ 
  $a_i \geq 0$
  とする.\ 
  このとき,\ 各
  $i$
  で,\ 
  $g_n \nearrow f$
  であるので,\ 
  $1_{A_i}g_n \nearrow 1_{A_i}f = a_i 1_{A_i}$
  となり,\ 
  $a_i^{-1}1_{A_i}g_n \nearrow 1_{A_i}$
  となる.\ 
  よって,\ 積分の線形性と(1)とより,\ 
  $\lim_n \mu(a_i^{-1}1_{A_i}g_n) = a_i^{-1} \lim_n \mu(1_{A_i}g_n) = \mu(A_i)$
  から,\ 
  $\lim_n \mu(1_{A_i}g_n) = a_i \mu(A_i)$
  となる.\ 
  また,\ 
  $g_n = \sum_{i=1}^m 1_{A_i}g_n$
  と表現できる.\ 
  したがって,\ 
  \begin{align}
    \lim_n \mu(g_n)
    & = \lim_{n \to \infty} \mu\left(\sum_{i=1}^m 1_{A_i}g_n \right) \\
    & = \lim_{n \to \infty} \sum_{i=1}^m \mu(1_{A_i}g_n) \\
    & = \sum_{i=1}^m \lim_{n \to \infty} \mu(1_{A_i}g_n) \\
    & = \sum_{i=1}^m a_i \mu(A_i) \\
    & = \mu(f).
  \end{align}
\end{proof}

\begin{lemma}[積分の唯一性]\label{lemma:int_unique}
  \mbox{}
  (1)
  $f \in m\mathcal{F}^+$
  とし,\ 
  二つの関数列
  $f^r,\ f_n \in SF^+$
  が各々
  $f^r \nearrow f,\ f_n \nearrow f$
  とする.\ 
  このとき,\ 
  $\lim_r \mu\left( f^r\right) = \lim_n \mu\left( f_n\right)$
  である.\ 

  (2)
  $f \in m\mathcal{F}^+$
  とし,\ 関数列
  $f_n \in SF^+$
  が
  $f_n \nearrow f$
  であるとする.\ 
  このとき,\ 
  $\lim_n \mu(f_n) = \mu(f)$
  である.\ 
\end{lemma}

\begin{proof}
  (1)
  $f_n^r:\Omega \to \mathbb{R} : \omega \mapsto \min\left( f^r(\omega), f_n(\omega)\right)$
  とする.\ 
  このとき,\ 
  $r$
  に関して
  $f_n^r\nearrow f_n$
  ,
  $n$
  に関して
  $f_n^r \nearrow f^r$
  となる.\ 
  補題\ref{lemma:point_convergence}(2)より,\ 
  $\lim_r \mu\left( f_n^r\right) = \mu(f_n)$,
  $\lim_n \mu\left( f_n^r\right) = \mu(f^r)$
  となり,\ 命題\ref{prop:doubly}から,\ 
  $\lim_r \mu\left( f^r\right) = \lim_n \mu\left( f_n\right)$
  である.\ 

  (2)
  非負値関数の定義より,\ 各点で
  $h_n \leq f $
  で
  $\lim_n \mu(h_n) = \mu(f)$
  となるような関数列
  $h_n \in SF^+$
  が存在する.\ 
  関数列
  $g_n \in SF^+$
  が
  $g_n \nearrow f$
  であるとする.\ 
  ここで,\ 
  $f'_n := \max(g_n,\ h_1,\ h_2,\ \ldots,\ h_n)$
  とすると,\ 
  $f'_n \in SF^+,\ f'_n \leq f$
  となり,\ 
  $f'_n \geq g$
  であるので,\ 
  $f'_n \nearrow f$
  である.\ 
  積分の定義から,\ 
  $\mu(f'_n) \leq \mu(f)$
  である.\ 
  さらに,\ 
  $f'_n \geq h_n$
  であるので,\ 
  $\mu(f'_n) \geq \mu(h_n)$
  である.\ 
  よって,\ 
  $\mu(f) = \lim_n \mu(h_n) \leq \lim_n \mu(f'_n)$
  となり,\ 
  $\lim_n \mu(f'_n) = \mu(f)$
  であることがわかった.\ 

  関数列
  $f_n\in SF^+$
  が
  $f_n \nearrow f$
  となるように任意に選んできたとき,\ (1)より,\ 
  $\lim_n \mu(f_n) = \lim_n \mu(f'_n) = \mu(f)$
  となる.\ 
\end{proof}

教科書の期待値の定義はこの補題のことを言っている.\ 

\begin{theorem}[単調収束定理]\label{theo:mono}
  $f_n \in m\mathcal{F}^+,\ f_n \nearrow f \ (\in m\mathcal{F}^+\footnote{可測関数の極限は可測関数である.\ })$
  とする.\ 
  このとき,\ 
  $\lim_n \mu(f_n) = \mu(f)$
  となる.\ 
\end{theorem}

\begin{proof}
  任意の
  $r \in \mathbb{N}$
  に対して,\ 
  \begin{equation}
    \alpha^r:\left[ 0,\infty\right] \to \left[ 0,\infty\right]:x\mapsto
    \begin{cases}
      0 & {\rm if}\ x=0 \\
      (i-1)2^{-r} & {\rm if}\ (i-1)2^{-r} < x \leq i2^{-r} \ (i\in\mathbb{N}) \\
      r & {\rm if} \ x > r,
    \end{cases}
  \end{equation}
  とする.\ 
  $f_n^r:=\alpha^r \circ f_n,\ f^r:=\alpha^r \circ f$
  とする.\ 
  明らかに
  $f_n^r,\ f^r \in SF^+$
  である.\ 
  $\alpha^r$
  は左連続関数で,\ 
  $f_n \nearrow f$
  だから,\ 
  $f_n^r \nearrow f^r$
  となるので,\ 補題\ref{lemma:point_convergence}(2)より,\ 
  $\lim_n \mu(f_n^r) = \mu(f^r)$
  である.\ 
  また,\ 
  $id:[0,\infty) \to [0,\infty)$
  を恒等写像とすると,\ 
  $a_r \nearrow id$
  (証明は\ref{subsec:proof}に記載)だから,\ 
  $f_n^r \nearrow f_n$
  となり,\ 補題\ref{lemma:int_unique}(2)より,\ 
  $\lim_r \mu(f_n^r) = \mu(f_n)$
  である.\ 
  さらに,\ 
  $f^r \nearrow f$
  であるので,\ 補題\ref{lemma:int_unique}(2)より,\ 
  $\lim_r \mu(f^r) = \mu(f)$
  である.\ 
  以上より,\ 命題\ref{prop:doubly}より,\ 
  $\lim_n \mu(f_n) = \lim_r \mu(f^r) = \mu(f)$
  とわかる.\ 
\end{proof}

\subsection{測度論の積分における主要な定理}
以下,\ 単調収束定理を用いて,\ 非負値可測関数の積分の性質,\ Fatouの補題,\ Lebesugueの収束定理を示す.\ 

\begin{lemma}[非負値関数の積分の性質]
  $f,\ g\in m\mathcal{F}^+$
  とする.\ 
  \begin{description}
    \item[(1)線形性]
      $\alpha,\ \beta \in \mathbb{R}$
      のとき,\ 
      $\mu(\alpha f + \beta g) = \alpha \mu(f) + \beta \mu(g)$
      となる.\ 
    \item[(2)単調性]
      $f \leq g$
      のとき,\ 
      $\mu(f) \leq \mu(g)$
      となる.\ 
  \end{description}
\end{lemma}

\begin{proof}
  $f_r := \alpha^r \circ f,\ g_r := \alpha^r \circ g$
  とすると,\ 
  $f_r,\ g_r \in SF^+$
  かつ,\ 
  $f_r \nearrow f,\ g_r \nearrow g$
  となる.\ 
  
  (1)
  単関数の積分の線形性より,\ 
  $\mu(\alpha f_r + \beta g_r) = \alpha \mu(f_r) + \beta \mu(g_r)$
  となる.\ 
  また,\ 明らかに
  $\alpha f_r + \beta g_r \nearrow \alpha f + \beta g$
  であるので,\ 単調収束定理より,\ 
  $\mu(\alpha f + \beta g_r) = \lim_r \mu(\alpha f_r + \beta g_r) = \lim_r \left(  \alpha \mu(f_r) + \beta \mu(g_r) \right) = \alpha \mu(f) + \beta \mu(g)$.

  (2)
  単関数の積分の単調性より,\ 
  $\mu(f_r) \leq \mu(g_r)$
  となる.\ 
  よって,\ 任意の
  $r \in \mathbb{N}$
  に対して,\ 
  $\mu(g_r - f_r) \geq 0$
  となる.\ 
  明らかに
  $g_r-f_r \nearrow g-f$
  であるので,\ 単調収束定理より,\ 
  $\mu(g-f) = \lim \mu(g_r - f_r) \geq 0$
  (1)より,\ 
  $\mu(g) \geq \mu(f)$
  となる.\ 
\end{proof}


\begin{lemma}[Fatouの補題]\label{lemma:Fatou}
  \mbox{}
  \begin{description}
    \item[(1)Fatouの補題]
      関数列
      $f_n \in m\mathcal{F}$
      は
      $\mu\left( \liminf_n f_n\right) \leq \liminf_n \mu(f_n)$
      となる.\ 
    \item[(2)逆Fatouの補題]
      関数列
      $f_n \in m\mathcal{F}$
      はある関数
      $g \in m\mathcal{F}$
      に対し
      $f_n \leq g$
      で,\ 
      $\mu(g) < \infty$
      とする.\ 
      このとき,\ 
      $\mu\left( \limsup_n f_n\right) \geq \limsup \mu(f_n)$
      となる.\ 
  \end{description}
\end{lemma}

\begin{proof}
  (1)
  関数列
  $(g_k)_{k=1}^{\infty}$
  を
  $g_k:= \inf_{n\leq k} f_n$
  とすると,\ 
  $\liminf_n f_n = \lim_k g_k$
  となる.\ 
  定義より
  $g_k$
  は単調増加なので,\ 単調収束定理より,\ 
  $\mu\left( \lim_k g_k\right) = \lim_k \mu(g_k)$
  である.\ 
  また,\ 
  $n \geq k$
  であるとき,\ 
  $f_n \geq g_k$
  であるので,\ 積分の単調性より
  $\mu(f_n) \geq \mu(g_k)$
  となり,\ 
  $\mu(g_k) \leq \inf_{n\geq k} \mu(f_n)$
  である.\ 
  以上より,\ 
  $\mu\left( \liminf_n f_n\right) = \mu\left( \lim_k g_k\right) = \lim_k \mu(g_k) \leq \lim_k \inf_{n\geq k} \mu(f_n) =: \liminf_n \mu(f_n)$
  である.\ 

  (2)
  $g-f_n \in m\mathcal{F}$
  に対し,\ (1)を適用すると
  $\mu\left( \liminf_n (g-f_n)\right) \leq \liminf_n \mu(g-f_n)$
  となる.\ 
  積分の線形性と
  $\liminf$
  の定義より,\ 
  $\mu\left( \liminf_n (g-f_n)\right) = \mu(g) - \mu\left( -\liminf_n -f_n\right) = \mu(g) - \mu\left( \limsup_n f_n\right)$,\ 
  $\liminf_n \mu(g-f_n) = \mu(g) + \liminf_n - \mu(f_n) = \mu(g) - \limsup_n \mu(f_n)$
  である.\ 
  以上の2式を比較すると,\ 
  $\mu(g) - \mu\left( \limsup_n f_n\right) \leq \mu(g) - \limsup_n \mu(f_n)$
  より,\ 
  $\mu\left( \limsup_n f_n\right) \leq \limsup_n \mu(f_n)$
  となる.\ 
\end{proof}

\begin{theorem}[Lebesugueの収束定理]
  $f_n,\ f \in m\mathcal{F}$
  とし,\ 
  $f_n$
  が
  $f$
  に各点収束するとする.\ 
  $g \in m\mathcal{F}^+$
  が可積分,\ 
  つまり
  $\mu(g) < \infty$
  で,\ かつ,\ 任意の
  $\omega \in \Omega,\ n \in \mathbb{N}$
  に対して
  $|f_n(\omega)| \leq g(s)$
  であるとする.\ 
  このとき,\ 
  $\lim_n \mu\left( |f_n - f|\right) = 0$
  となり,\ 
  $\lim_n \mu(f_n) = \mu(f)$
  となる.\ 
\end{theorem}

\begin{proof}
  $|f_n-f| \leq 2g$
  かつ,\ 
  $\mu(2g) < \infty$
  であるから,\ 逆Fatouの補題より,\ 
  $\limsup_n \mu\left( |f_n-f|\right) \leq \mu\left( \limsup_n |f_n - f|\right) = \mu(0) = 0$
  となる.\ 
  最後から2番目の式に出てくる
  $0$
  は常に
  $0$
  を返す関数である.\ 

  したがって,\ 
  $|\mu(f_n) - \mu(f)| = |\mu(f_n - f)| \leq \mu(|f_n - f|)$
  より,\ 
  $0 \leq \limsup_n |\mu(f_n - f)| \leq \limsup_n \mu(|f_n-f|) = 0 $
  であるので,\ 
  $\lim_n |\mu(f_n - f)| = 0$
  となる.\ 
  したがって,\ 
  $\lim_n \mu(f_n) = \mu(f)$
  となる.\ 
\end{proof}

\subsection{$\alpha_r \nearrow id$の証明}\label{subsec:proof}
\begin{proof}
  任意の
  $x\in \left[ 0, \infty \right)$,\ 
  $\epsilon > 0$
  に対して,\ 
  $r_0 = \lceil \max(x, -\log_2 \epsilon) \rceil$
  とする.\ 
  $r > r_0$
  とする.\ 
  $x \leq r_0 < r$
  だから,\ ある
  $i \in \mathbb{N}$
  が存在して
  $(i-1)2^{-r} < x \leq i2^{-r}$
  とできる.\ 
  よって,\ 
  $i \geq x2^r$
  であるので,\ 
  $0 < x - \alpha^r(x) = x - (i-1)2^{-r} \leq x - (x2^r - 1)2^{-r} = 2^{-r}$
  となる.\ 
  $- \log_2 \epsilon < r$
  だから,\ 
  $2^{-r} < \epsilon$
  となる.\ 
  したがって,\ 
  $0 < x- \alpha^r(x) < \epsilon$
  を示すことができたので,\ 
  $\alpha^r \nearrow id$.
\end{proof}
%\bibliographystyle{junsrt}
%\bibliography{}
\end{document}
